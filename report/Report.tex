\documentclass[a4paper]{report}
\usepackage[utf8]{inputenc}
\usepackage[portuguese]{babel}
\usepackage{hyperref}
\usepackage{a4wide}
\hypersetup{pdftitle={Trabalho 1},
pdfauthor={João Teixeira, José Ferreira, Miguel Solino},
colorlinks=true,
urlcolor=blue,
linkcolor=black}
\usepackage{subcaption}
\usepackage[cache=false]{minted}
\usepackage{listings}
\usepackage{booktabs}
\usepackage{multirow}
\usepackage{appendix}
\usepackage{tikz}
\usepackage{authblk}
\usepackage{bashful}
\usepackage{verbatim}
\usepackage{amsmath}
\usetikzlibrary{positioning,automata,decorations.markings}

\begin{document}

\title{Fase 1\\ 
\large Grupo Nº 8}
\author{João Teixeira (A85504) \and José Ferreira (A83683) \and Maria Silva (A83840) \and Miguel Solino (A86435)}
\date{\today}

\begin{center}
    \begin{minipage}{0.75\linewidth}
        \centering
        \includegraphics[width=0.4\textwidth]{images/eng.jpeg}\par\vspace{1cm}
        \vspace{1.5cm}
        \href{https://www.uminho.pt/PT}
        {\color{black}{\scshape\LARGE Universidade do Minho}} \par
        \vspace{1cm}
        \href{https://www.di.uminho.pt/}
        {\color{black}{\scshape\Large Departamento de Informática}} \par
        \vspace{1.5cm}
        \maketitle
    \end{minipage}
\end{center}

\tableofcontents

\pagebreak

\chapter{Introdução}

Este relatório tem como objetivo apresentar o que foi realizado para esta fase
inicial do projeto da Unidade Curricular de Desenvolvimento de Sistemas de
Software. Ao longo do que é apresentado no documento procuramos justificar
todas as considerações realizadas na formulação e elaboração dos modelos e
diagramas: Modelo de Domínio, Use Cases, Diagrama de Use Cases, Especificações
dos mesmos e mockups.

\chapter{Modelo de Domínio}

O sistema a implementar deve suportar a gestão de media e utilização da mesma
para uma residência onde esteja instalado.

Neste capítulo são apresentados os requisitos do problema e uma proposta de 
Modelo de Domínio.

\section{Descrição do Modelo de Domínio}

O problema que nos foi atribuído é desenvolver um sistema de gestão e streaming
de media. Depois de uma análise do que era necessário para a realização deste,
chegamos à conclusão que era necessário a existência dos seguintes conceitos, a
incluir na modelação do problema: Utilizador, Media (música e vídeo),
Biblioteca de media, Caterização (da media que for adicionada ao sistema por
um utilizador), Download/Upload de media, Conta, Conexões (de amizades entre utilizadores, ou seja, uma lista de Amigos).

Com isto, foram definidas as relações entre os conceitos referidos em cima:
\begin{itemize}
    \item cada Utilizador só pode ter uma conta, só pode fazer download da
        media da qual fez upload e só pode classificar a media e bibliotecas
        que lhe pertencem;
    \item cada media/biblioteca de media pode ter uma ou mais classificações;
    \item cada conta tem só um email, nome e password associados;
    \item conta não é obrigatório para utilizar o sistema mas fica limitado 
        a só usufruir da media;
\end{itemize}
Os utilizadores não devem ter a permissão de criar contas, logo para um
utilizador ter uma conta precisa que algum outro com permissões de
administrador crie a conta sem password para mais tarde os utilizadores
das contas a definam.

\section{Modelo de Domínio}

\chapter{Use Cases}

O Diagrama de Use Cases representa os atores do sistema e as tarefas de cada um
deles. No nosso caso existem 3 atores:
\begin{itemize}
    \item \textbf{Utilizador}: Todos os outros referidos abaixo são atores deste tipo.
        Logo, a este ator estará relacionado as ações que todos podem fazer
    \item \textbf{Utilizador sem password}: É um utilizador que após ter pedido a um
        administrador para lhe criar uma conta dando o email e nome, pode definir a 
        sua password para passar a ser um utilizador com password
    \item \textbf{Utilizador com password}: Ter password significa que já tem uma conta 
        "completa", ou seja, tem acesso à totalidade de ações que um user sem permissões
        de administrador pode ter
    \item \textbf{Utilizador com permissões de administrador}: A diferença deste 
        utilizador para os restantes é que este é o único que pode criar uma conta 
        e assim ser o responsável pela existência dos restantes.
\end{itemize}

\section{Especificações dos Use Cases}

Nesta parte optamos por ordenar os use cases por permissões, ou seja, começamos
por apresentar os de um utilizador normal, seguido de um utilizador sem password, utilizador com password e utilizador com permissões de administrador.

\subsection{Entrar como convidado}

\subsection{Media Control}

\subsection{Escolha de media}

\subsection{Definir password}

\subsection{Login}

\subsection{Logout}

\subsection{Download}

\subsection{Upload}

\subsection{Classificar}

\subsection{Criar Bibliotecas}

\subsection{Criar Conta}

\chapter{Mockups}

Mockups é o nome dado aos protótipos de interface e como o próprio nome indica
é uma demonstração de como a aplicação final vai ser.

\end{document}
